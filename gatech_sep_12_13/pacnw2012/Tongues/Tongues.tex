\documentclass{article}

\usepackage{geometry}
\usepackage{verbatim}
\usepackage{tabularx}

\title{Tongues}
\date{}

\begin{document}
\maketitle

Gandalf's writings have long been available for study, but no one
has yet figured out what language they are written in.  Recently,
due to programming work by a hacker known only by the code name
ROT13, it has been discovered that Gandalf used nothing but a
simple letter substitution scheme, and further, that it is its
own inverse---the same operation scrambles the message as
unscrambles it.

This operation is performed by replacing vowels in the sequence
\begin{quote}
(a i y e o u)
\end{quote}
with the vowel three advanced, cyclicly, while preserving case (i.e., lower or
upper).  Similarly, consonants
are replaced from the sequence
\begin{quote}
(b k x z n h d c w g p v j q t s r l m f)
\end{quote}
by advancing ten letters.  So for instance the phrase
\begin{quote}
   One ring to rule them all.
\end{quote}
translates to
\begin{quote}
   Ita dotf ni dyca nsaw ecc.
\end{quote}
The fascinating thing about this transformation is that the resulting
language yields pronounceable words.

For this problem, you will write code to translate Gandalf's
manuscripts into plain text.

\section{Input}

The input file will contain multiple test cases.  Each test case consists of a
single line containing up to 100 characters, representing some text written
by Gandalf.  All characters will be plain ASCII, in the range space (32)
to tilde (126), plus a newline terminating each line.
The end of the input is denoted by the end-of-file.

\section{Output}

For each input test case, print its translation into plaintext.
The output should contain exactly the same number of
lines and characters as the input.

\vskip 16pt
\noindent
\setlength{\extrarowheight}{4pt}
\begin{tabularx}{\textwidth}{ | X | X | }
\hline
\textbf{Sample Input} & \textbf{Sample Output} \\
\verbatiminput{Tongues.sample.in}
&
\verbatiminput{Tongues.sample.out}
\\
\hline
\end{tabularx}

\end{document}
